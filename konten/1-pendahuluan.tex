\section{PENDAHULUAN}

\subsection{Latar Belakang}

% Ubah paragraf-paragraf berikut sesuai dengan latar belakang dari tugas akhir
Pramuka merupakan singkatan dari Praja Muda Karana yang berbentuk dalam sebuah Organisasi atau Gerakan Kepanduan yang di Indonesia sendiri biasa disebut dengan Gerakan Pramuka . Di Dalam Dunia Internasional dikenal dengan nama Kepanduan atau biasa disebut Boy Scout . Gerakan Pramuka memiliki Kode Kehormatan yang terdiri dari janji yang disebut dengan Satya dan Ketentuan Moral . Pramuka sendiri dibagi menjadi beberapa golongan yaitu Siaga , Penggalang , Penegak dan Pandega 
 

 Semaphore merupakan sebuah cara berkomunikasi satu dengan lainnya yang saling berjauhan dengan dua tangan memegang bendara . Dari pose yang diperagakan maka bisa diterima informasi tentang huruf apa yang diperagakan . Semaphore merupakan materi yang diajarkan didalam materi Pramuka di setiap sekolah-sekolah . Manfaat mempelajari semaphore sendiri adalah agar setiap anggota Pramuka bisa berkomunikasi satu sama lain di situasi susah sinyal 

 Permasalahan yang dihadapi dalam Semaphore dalam Dunia Pendidikan sendiri adalah masih banyak anak-anak yang kesulitan dalam memahami Gerakan-gerakan semaphore dalam pembelajaran Pramuka , Penelitian tentang pemanfaatan teknologi visi computer yaitu tensor flow dalam sistem deteksi tubuh manusia


\subsection{Rumusan Masalah}

% Ubah paragraf berikut sesuai dengan rumusan masalah dari tugas akhir
Saat ini sedikit sekali orang yang bisa menguasai dan membaca huruf semaphore , Oleh karena itu diperlukan sebuah Teknologi yang dapat membantu untuk menyelesaikan persoalan ini yaitu dengan cara menggunakan pendekatan Deep Learning 


\subsection{Batasan Masalah atau Ruang Lingkup}

Berdasarkan dari tujuan penelitian ini, diperlukan adanya batasan-batasan masalah untuk memperjelas cakupan dari penelitian yang dilakukan. Adapun batasan-batasan tersebut sebagai berikut:

\begin{enumerate}
\item Huruf Yang Dideteksi Sebanyak 10 Huruf
\item Huruf Yang Dideteksi Adalah Huruf Alphabet 
\end{enumerate}


\subsection{Tujuan}


% Ubah paragraf berikut sesuai dengan tujuan penelitian dari tugas akhir


Tujuan dari penelitian ini adalah :

\begin{enumerate}   
\item	Menerapkan deteksi pose manusia menggunakan MediaPipe kepada tangan
\item	Menerjemahkan gestur semaphore pramuka menjadi audio Bahasa Indonesia
\item	Memberikan pengalaman pembelajaran interaktif tentang semaphore
\end{enumerate}


\subsection{Manfaat}

% Ubah paragraf berikut sesuai dengan tujuan penelitian dari tugas akhir
Manfaat dari penelitian ini adalah : 

\begin{enumerate}  
\item	Bagi penulis manfaat yang akan didapat adalah menambah ilmu dalam visi komputer dan Deep Learning dalam penggunaan MediaPipe.
\item	Bagi Institusi manfaat yang akan didapat adalah bentuk pengimplementasian yang nyata terhadap ilmu visi komputer.
\item	Bagi peneliti manfaat yang didapat adalah dapat mengembangkan model machine learning dengan menggunakan MediaPipe dan juga model Deep Learning.
\item	Bagi Siswa yang masih sekolah bisa bermanfaat untuk membantu dalam proses pembelajaran semaphore yang dilaksanakan didalam kegiatan pramuka
\end{enumerate}

