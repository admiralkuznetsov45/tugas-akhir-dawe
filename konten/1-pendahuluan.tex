\chapter{PENDAHULUAN}

\section{Latar Belakang}

% Ubah paragraf-paragraf berikut sesuai dengan latar belakang dari tugas akhir
Semaphore adalah sistem komunikasi visual yang menggunakan bendera atau tangan saat mengirim pesan \cite{howeth1963history}. Sistem ini telah digunakan secara luas dalam militer, kapal laut, dan sektor transportasi dengan mengirim pesan antara dua atau lebih pihak. Namun, penggunaan semaphore secara tradisional membutuhkan latihan dan keahlian khusus, serta keterampilan memahami kode semaphore yang kompleks \cite{zhao2016convolutional}.

Dalam era digital saat ini, teknologi pengenalan pose tubuh telah berkembang pesat. Salah satu teknologi tersebut adalah \textit{MediaPipe} yang dikembangkan oleh \textit{Google} \cite{huynh2020mediapipe}. MediaPipe adalah kerangka kerja pengenalan pose tubuh yang memungkinkan penggunaan teknologi pengenalan pose secara \textit{real-time}. Teknologi ini memiliki berbagai aplikasi dalam industri \textit{virtual reality}\cite{huynh2020mediapipe}, \textit{augmented reality} \cite{lugaresi2019mediapipe}, dan interaksi manusia-komputer. \cite{harris2021applying}

Dalam konteks ini, prediksi huruf semaphore berbasis pose \textit{MediaPipe} dapat menjadi langkah inovatif  mengatasi kendala yang terkait dengan komunikasi semaphore tradisional. Dengan memanfaatkan teknologi pengenalan pose tubuh, komunikasi menggunakan semaphore dapat lebih mudah dan dapat diakses oleh lebih banyak orang tanpa perlu melalui pelatihan khusus. \cite{singh2022realtime}
Deep Learning adalah  bagian dari dari \textit{Machine Learning} yang memanfaatkan jaringan saraf tiruan dengan berbagai tingkatan kedalaman. Jaringan saraf ini mampu belajar dari sejumlah besar data yang tidak terstruktur atau setengah terstruktur dengan cara yang sangat efisien. \cite{deng2014deep}

Jaringan saraf dalam \textit{Deep Learning} disebut sebagai \textit{"deep"} karena mereka memiliki banyak lapisan. Setiap lapisan dalam jaringan ini mengubah input yang diterima lalu mendapatkan representasi yang lebih abstrak dan kompleks. Lapisan-lapisan ini memungkinkan komputer belajar secara langsung dari data mentah seperti gambar, teks, atau suara, yang merupakan perbedaan utama dengan metode tradisional yang biasanya membutuhkan pengkodean fitur manual.

\textit{CNN (Convolutional Neural Network)} adalah salah satu metode yang populer dan efektif dalam mempelajari pola visual dari data gambar \cite{lecun2015deep}. CNN telah terbukti sukses dalam berbagai aplikasi pengenalan citra dan pengenalan pola lainnya \cite{kim2023human}. Dengan menggunakan metode ini, prediksi huruf semaphore dapat dilakukan dengan mengolah data pose tubuh yang diperoleh dari \textit{MediaPipe}. 

Dengan adanya prediksi huruf semaphore berbasis pose menggunakan \textit{MediaPipe} dan \textit{Deep Learning}, komunikasi melalui semaphore dapat menjadi lebih efisien dan dapat diakses oleh lebih banyak orang. Tidak diperlukan lagi latihan khusus atau pemahaman kode semaphore yang rumit, karena teknologi ini memungkinkan komputer secara otomatis mengenali dan menerjemahkan pose tubuh menjadi huruf-huruf semaphore. Ini akan membuka peluang baru dalam meningkatkan efektivitas komunikasi dalam situasi yang membutuhkan metode komunikasi visual seperti di lapangan militer, kapal laut, atau sektor transportasi.


\section{Rumusan Masalah}
Saat ini sedikit sekali orang yang bisa menguasai dan membaca huruf semaphore , Oleh karena itu diperlukan sebuah Teknologi yang dapat membantu  menyelesaikan persoalan ini yaitu dengan cara menggunakan pendekatan \textit{Deep Learning} melalui implementasi \textit{Convolutional Neural Network} dengan bantuan deteksi fitur tulang yang disediakan oleh \textit{MediaPipe} dalam rangka membantu deteksi agar bisa meningkatkan akurasi dan efisiensi waktu dalam rangka mendukung mempelajari semaphore 

\section{Batasan Masalah}
Batasan masalah dari penelitian antara lain; Pose tubuh yang dideteksi hanya badan sampai lengan dan pergelangan tangan , huruf yang dideteksi menggunakan pose tubuh , huruf yang dideteksi menggunakan huruf alfabet dan metode  \textit{Deep Learning} yang diaplikasikan prediksi semaphore berbasis poster tubuh adalah \textit{Convolutional Neural Network}
 

\section{Tujuan}
Tujuan dari penelitian ini adalah melakukan prediksi pose Semaphore yang memiliki sumber dari dataset bagian atas tubuh manusia dengan latar belakang hitam sebagai input dalam proses training dan membuat kata dari pose semaphore yang diperagakan lalu bisa mendapatkan perbandingan dari masing-masing model yang terdiri dari Model CNN , Model CNN2 , Model CNNResNet50V2 dan Model Xception . 

\section{Manfaat}
Manfaat dari penelitian ini adalah melakukan prediksi pose Semaphore dan juga membentuk kata dari pose yang diperagakana . Teknologi ini bermanfaat para peserta didik yang masih duduk di bangku sekolah