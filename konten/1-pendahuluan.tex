\section{PENDAHULUAN}

\subsection{Latar Belakang}

% Ubah paragraf-paragraf berikut sesuai dengan latar belakang dari tugas akhir
Bendera Semaphore adalah sebuah sistem komunikasi yang menggunakan posisi dan orientasi bendera untuk mengirimkan pesan. Bendera Semaphore sering digunakan di lautan dan di tempat-tempat yang jauh dari pemukiman manusia, seperti di pulau-pulau terpencil atau di tengah-tengah hutan belantara

Meskipun tidak lagi digunakan secara luas seperti sebelumnya, Bendera Semaphore dapat memberikan beberapa manfaat secara luas seperti  masih bisa digunakan sebagai salah satu cara komunikasi darurat atau sebagai keterampilan yang berguna dalam kegiatan outdoor atau komunitas militer. Lalu Semaphore bisa bermanfaat bisa untuk membantu seseorang meningkatkan kemampuan observasi, karena pengguna harus mampu memperhatikan dengan seksama posisi dan orientasi bendera yang digunakan untuk mengirimkan pesan

Selanjunya Mempelajari Semaphore bisa untuk  membantu seseorang meningkatkan kemampuan berpikir dan problem solving, karena pengguna harus mampu memecahkan masalah dan mengambil keputusan dengan cepat dan tepat berdasarkan pesan yang diterima melalui bendera. 

Semaphore juga bisa meningkatkan kemampuan kerja sama, karena sistem ini membutuhkan beberapa orang untuk bekerja sama dan saling bergantung satu sama lain dalam mengirimkan dan menerima pesan

Permasalahan yang dialami saat ini adalah orang-orang saat ini tidak banyak yang menguasai cara berkomunikasi menggunakan bendera semaphore karena menguasai cara berkomunikasi menggunakan semaphore memerlukan kemampuan untuk bisa menghafal pose-pose huruf yang ada dalam bendera semaphore . 

Pada penelitian ini penulis memanfaatkan kemampuan teknologi Deep Learning yaitu implementasi Convolutional Neural Network dan juga menggunakan fitur estimasi pose tubuh manusia dalam implementasinya yaitu membuat sistem prediksi huruf semaphore . Contoh dari implementasi Deep Learning sudah banyak digunakan di berbagai aspek kehidupan manusia seperti fitur analisis gambar dan video untuk melakukan identifikasi orang atau objek yang tidak diizinkan masuk kedalam suatu area . Diharapkan Penelitian ini bisa membantu dan memudahkan dalam mempelajari bendera semaphore

\subsection{Rumusan Masalah}

% Ubah paragraf berikut sesuai dengan rumusan masalah dari tugas akhir
Saat ini sedikit sekali orang yang bisa menguasai dan membaca huruf semaphore , Oleh karena itu diperlukan sebuah Teknologi yang dapat membantu untuk menyelesaikan persoalan ini yaitu dengan cara menggunakan pendekatan Deep Learning 


\subsection{Batasan Masalah atau Ruang Lingkup}

Berdasarkan dari tujuan penelitian ini, diperlukan adanya batasan-batasan masalah untuk memperjelas cakupan dari penelitian yang dilakukan. Adapun batasan-batasan tersebut sebagai berikut:

\begin{enumerate}
\item Huruf Yang Dideteksi Adalah Huruf Alphabet 
\end{enumerate}


\subsection{Tujuan}


% Ubah paragraf berikut sesuai dengan tujuan penelitian dari tugas akhir


Tujuan dari penelitian ini adalah :

\begin{enumerate}   
\item	Menerapkan deteksi pose manusia menggunakan MediaPipe kepada tangan
\item	Menerjemahkan gestur semaphore pramuka menjadi audio Bahasa Indonesia
\item	Memberikan pengalaman pembelajaran interaktif tentang semaphore
\end{enumerate}


\subsection{Manfaat}

% Ubah paragraf berikut sesuai dengan tujuan penelitian dari tugas akhir
Manfaat dari penelitian ini adalah : 

\begin{enumerate}  
\item	Bagi penulis manfaat yang akan didapat adalah menambah ilmu dalam visi komputer dan Deep Learning dalam penggunaan MediaPipe.
\item	Bagi Institusi manfaat yang akan didapat adalah bentuk pengimplementasian yang nyata terhadap ilmu visi komputer.
\item	Bagi peneliti manfaat yang didapat adalah dapat mengembangkan model machine learning dengan menggunakan MediaPipe dan juga model Deep Learning.
\item	Bagi Siswa yang masih sekolah bisa bermanfaat untuk membantu dalam proses pembelajaran semaphore yang dilaksanakan didalam kegiatan pramuka
\end{enumerate}