
\chapter{KESIMPULAN DAN SARAN}

\section{Kesimpulan}
Berdasarkan hasil dan pembahasan terkait empat Model \textit{CNN} dalam tugas Prediksi Pose Semaphore Berbasis Deep Learning dapat diperoleh beberapa kesimpulan tentang kinerja masing-masing model. Selain kinerja masing-masing model juga dibahas tentang pengujian dilakukan dengan membandingkan performa Model \textit{CNN}, Model \textit{CNN2}, \textit{ResNet50V2}, dan \textit{Xception} dalam mengenali pose semaphore yang dihasilkan oleh sembilan koresponden berbeda dengan lima variasi gerakan dari huruf A hingga E. Berikut ini kesimpulan yang bisa diperoleh 

1. Perbandingan hasil \textit{loss} dari keempat model (Model CNN, Model \textit{CNN2}, \textit{ResNet50V2}, dan \textit{Xception}) pada Gambar \ref{fig:GrafikPerbandinganLoss} memberikan gambaran kuantitatif tentang performa dan konvergensi masing-masing model selama proses pelatihan. Semakin rendah nilai loss, semakin baik model tersebut dalam menemukan representasi yang tepat dari data dan mengoptimalkan parameter dalam rangka tugas pengenalan pose semaphore 
\begin{figure}[!hbt]
	\centering
	\includegraphics[width=0.7\linewidth]{gambar/bener/Perbandingan_LossCNN.png}
	\captionof{figure}{Grafik Perbandingan Loss}
	\label{fig:GrafikPerbandinganLoss}
\end{figure}
Model \textit{CNN} dan Model \textit{CNN2} menunjukkan hasil yang lebih baik dibandingkan dengan ResNet50V2 dan \textit{Xception}. Dari hasil \textit{loss} yang terekam selama 30 \textit{epoch}, Model \textit{CNN} dan Model \textit{CNN2} mencapai nilai \textit{loss} terendah lebih cepat dan secara konsisten memiliki nilai \textit{loss} yang lebih rendah dibandingkan dua model lainnya.

Model \textit{CNN} mencapai nilai \textit{loss} terendah sekitar 0.0044 pada \textit{epoch} ke-30, sedangkan Model \textit{CNN2} mencapai nilai \textit{loss} terendah sekitar 0.0043 pada \textit{epoch} ke-29. Dalam kasus ini, Model \textit{CNN} dan Model \textit{CNN2} memiliki hasil \textit{loss} yang sangat mendekati satu sama lain, menandakan kualitas representasi yang serupa dalam tugas pengenalan pose semaphore.

Sementara itu, \textit{ResNet50V2} dan \textit{Xception} juga menunjukkan penurunan nilai \textit{loss} yang konsisten selama proses pelatihan. Namun, keduanya memiliki nilai \textit{loss} yang agak lebih tinggi dibandingkan dua model sebelumnya. ResNet50V2 mencapai nilai \textit{loss} terendah sekitar 0.0083 pada \textit{epoch} ke-30, sementara \textit{Xception} mencapai nilai \textit{loss} terendah sekitar 0.0065 pada \textit{epoch} ke-26.

2. Berdasarkan hasil perbandingan akurasi dari empat model sesuai dengan Gambar \ref{fig:GrafikPerbandinganAkurasi}, yaitu Model \textit{CNN}, Model \textit{CNN2}, \textit{ResNet50V2}, dan \textit{Xception}, pada \textit{epoch} terakhir yaitu epich ke-30 . terlihat bahwa Model \textit{CNN2} memiliki akurasi tertinggi pada epoch terakhir, yaitu sekitar 93.53\%. Model ini juga memiliki validasi akurasi yang cukup tinggi, mencapai sekitar 97.20\%. Model \textit{CNN} mengalami performa yang hampir setara dengan Model \textit{CNN2}, dengan akurasi pada epoch terakhir sekitar 93.72\%, dan validasi akurasi sekitar 97.50\%.
\begin{figure}[!hbt]
	\centering
	\includegraphics[width=0.7\linewidth]{gambar/bener/Perbandingan_AkurasiCNN.png}
	\captionof{figure}{Grafik Perbandingan Loss}
	\label{fig:GrafikPerbandinganAkurasi}
\end{figure}
Sementara itu, Model \textit{ResNet50V2} dan Model \textit{Xception} memiliki akurasi yang sedikit lebih rendah daripada kedua model \textit{CNN} tersebut. Model \textit{ResNet50V2} mencapai akurasi sekitar 88.36\%, dengan validasi akurasi sekitar 96.59\%. Sedangkan, Model \textit{Xception} memiliki akurasi dan validasi akurasi yang sama dengan Model \textit{ResNet50V2}, yaitu sekitar 88.36\% dan 92.39\% pada epoch terakhir.

Dari hasil ini, dapat disimpulkan bahwa Model \textit{CNN2} adalah yang memiliki akurasi tertinggi pada epoch terakhir, diikuti oleh Model CNN. Meskipun Model \textit{ResNet50V2} dan Model \textit{Xception} memiliki akurasi yang lebih rendah, namun keduanya masih menunjukkan kinerja yang cukup baik dalam tugas klasifikasi pose semaphore.

3. Berdasarkan Hasil Evaluasi Metrik seperti pada Gambar \ref{fig:GrafikPerbandinganEvaluasiMetrik} dari \textit{Precision} , \textit{Recall} dan \textit{F1-Score} dari masing-masing model dapat diambil hasil yaitu Model \textit{CNN} menunjukkan tingkat \textit{precision} sebesar 0.973, \textit{recall} sebesar 0.809, dan \textit{F1-Score} sebesar 0.881. Sementara itu, model \textit{CNN2} mencapai \textit{precision} sebesar 0.968, \textit{recall} sebesar 0.946, dan \textit{F1-Score} sebesar 0.952, menunjukkan performa yang sangat baik dalam mengenali dan memprediksi kelas data.
\begin{figure}[!hbt]
	\centering
	\includegraphics[width=0.7\linewidth]{gambar/bener/RataRataEvaluasiMetrik.png}
	\captionof{figure}{Hasil Rata-Rata Evaluasi Metrik dari masing-masing model CNN}
	\label{fig:GrafikPerbandinganEvaluasiMetrik}
\end{figure}
Selanjutnya, model \textit{CNN} \textit{ResNet50V2} menunjukkan hasil evaluasi yang sangat mengesankan, dengan \textit{precision} mencapai 0.974, \textit{recall} sebesar 0.978, dan \textit{F1-Score} sebesar 0.975. Hal ini menandakan bahwa model \textit{CNN} ResNet50V2 memiliki kemampuan yang sangat tinggi dalam mengenali dan mengklasifikasikan data dengan akurasi tinggi. Terakhir, model \textit{CNN} \textit{Xception} juga menunjukkan performa yang baik, dengan \textit{precision} sebesar 0.959, \textit{recall} sebesar 0.961, dan \textit{F1-Score} sebesar 0.960.

\section{Saran}
Penelitian ini merupakan penelitian dasar yang dapat dikembangkan lebih lanjut diantaranya dengan  memperbanyak dataset huruf ,	Menambahkan variasi latar dan objek yang berbeda pada saat pengumpulan data , lebih teliti dalam memasukkan data pelatihan serta memperbaiki dalam kemampuan deteksi huruf