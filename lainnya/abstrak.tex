\chapter*{ABSTRAK}
\begin{center}
  \large
  \textbf{POSE BASED SEMAPHORE PREDICTION USING DEEP LEARNING \emph{DEEP LEARNING}}
\end{center}
\addcontentsline{toc}{chapter}{ABSTRAK}
% Menyembunyikan nomor halaman
\thispagestyle{empty}

\begin{flushleft}
  \setlength{\tabcolsep}{0pt}
  \bfseries
  \begin{tabular}{ll@{\hspace{6pt}}l}
  Nama Mahasiswa / NRP&:& Muhammad Daffa ZW / 0721 19 4000 0064\\
  Departemen&:& Teknik Komputer - FTEIC - ITS \\
  Dosen Pembimbing&:& 1. Dr Eko Mulyanto Yuniarno , S.T , M.T.\\
  & & 2. Ahmad Zaini, S.T., M.T.\\
  \end{tabular}
  \vspace{4ex}
\end{flushleft}
\textbf{Abstrak}

% Isi Abstrak
Penelitian ini bertujuan untuk mengembangkan sistem prediksi huruf semaphore menggunakan teknologi deep learning dengan implementasi convolutional neural network (CNN). Batasan masalah dari penelitian ini adalah hanya huruf alfabet yang dapat dideteksi menggunakan pose tubuh, dan metode deep learning yang diaplikasikan adalah CNN. Tujuan dari penelitian ini adalah untuk menerapkan deteksi pose manusia menggunakan MediaPipe kepada tangan, menerjemahkan gestur semaphore pramuka menjadi audio Bahasa Indonesia, serta memberikan pengalaman pembelajaran interaktif tentang semaphore. Penelitian ini diharapkan dapat membantu dan memudahkan dalam mempelajari bendera semaphore, meskipun penggunaannya saat ini telah berkurang.

\vspace{2ex}
\noindent
\textbf{Kata Kunci: \emph{Bendera Semaphore, Deep Learning, Convolutional Neural Network, Estimasi Pose Tubuh Manusia, MediaPipe, Pembelajaran Interaktif.}}