\chapter*{ABSTRAK}
\begin{center}
  \large
	  \textbf{PREDIKSI HURUF SEMAPHORE BERBASIS POSE MENGGUNAKAN emph{DEEP LEARNING}}
\end{center}
\addcontentsline{toc}{chapter}{ABSTRAK}
% Menyembunyikan nomor halaman
\thispagestyle{empty}

\begin{flushleft}
  \setlength{\tabcolsep}{0pt}
  \bfseries
  \begin{tabular}{ll@{\hspace{6pt}}l}
  Nama Mahasiswa / NRP&:& Muhammmad Daffa ZW / 07211940000064\\
  Departemen&:& Teknik Komputer FTEIC - ITS\\
  Dosen Pembimbing&:& 1. Dr. Eko Mulyanto Yuniarno, S.T., M.T.\\
  & & 2.  Ahmad Zaini, S.T., M.T.\\
  \end{tabular}
  \vspace{4ex}
\end{flushleft}
\textbf{Abstrak}

% Isi Abstrak

Sistem komunikasi visual menggunakan semaphore telah lama digunakan dalam berbagai bidang, seperti militer, kapal laut, dan sektor transportasi. Metode ini melibatkan penggunaan bendera atau tangan mengirim pesan antara dua atau lebih pihak. Namun, penggunaan semaphore secara tradisional membutuhkan latihan dan keahlian khusus, serta kemampuan memahami kode semaphore yang kompleks.Penelitian ini bertujuan mengembangkan sistem rekognisi pose semaphore menggunakan MediaPipe dan klasifikasi Deep Learning melalui metode Convolutional Neural Network (CNN). Proses pengembangan sistem melibatkan langkah-langkah seperti pengumpulan dan pengolahan dataset, ekstraksi pose menggunakan MediaPipe, dan klasifikasi pose menggunakan model-model CNN seperti CNN, CNN ResNet50V2, dan CNN Xception.Hasil penelitian ini menunjukkan bahwa sistem yang dikembangkan mampu mendeteksi huruf dan kata dalam pose semaphore dengan tingkat akurasi yang tinggi. Model CNN ResNet50V2 mencapai akurasi 88.36\%, Model CNN Xception mencapai akurasi 88.36\%, Model CNN mencapai akurasi 93.72\%, dan Model CNN2 mencapai akurasi 93.53\%.Berdasarkan temuan ini, sistem rekognisi pose semaphore memiliki potensi pemanfaatan dalam pendidikan di sekolah. Dengan bantuan teknologi ini, pembelajaran tentang semaphore dapat lebih menarik dan interaktif. Siswa dapat belajar mengenai komunikasi visual melalui semaphore dengan bantuan sistem yang mampu mendeteksi dan mengenali pose huruf dan kata. Penggunaan sistem rekognisi pose semaphore dalam pendidikan juga dapat memberikan kemungkinan eksplorasi lebih lanjut dalam pengembangan metode pembelajaran interaktif lainnya

\vspace{2ex}
\noindent
\textbf{Kata Kunci: \emph{Semaphore, MediaPipe, Convolutional Neural Network , Pose Tubuh, Deep Learning , Xception , ResNet50V2 }}