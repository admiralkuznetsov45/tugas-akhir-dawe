\begin{center}
    \Large
    \textbf{KATA PENGANTAR}
  \end{center}
  
  \addcontentsline{toc}{chapter}{KATA PENGANTAR}
  
  \vspace{2ex}
  
  % Ubah paragraf-paragraf berikut dengan isi dari kata pengantar
  
  Puji syukur kami panjatkan ke hadirat Allah SWT atas segala rahmat, hidayah, dan karunia-Nya sehingga penulis dapat menyelesaikan skripsi dengan judul "Prediksi Pose Semaphore Berbasis MediaPipe menggunakan Deep Learning." sebagI syarat mendapatkan gelar Sarjana Teknik Komputer

  Skripsi ini merupakan hasil perjuangan dan kerja keras penulis dalam mengeksplorasi pengetahuan dan pengalaman yang telah diperoleh selama masa studi di Institut Teknologi Sepuluh Nopember. Penulisan skripsi ini bertujuan untuk memenuhi salah satu syarat kelulusan dan mendapatkan gelar Sarjana Teknik Komputer.

  Dalam kesempatan ini, penulis ingin mengucapkan terima kasih yang sebesar-besarnya kepada semua pihak yang telah memberikan dukungan, bantuan, dan motivasi sehingga skripsi ini dapat terselesaikan dengan baik:
  
  \begin{enumerate}[nolistsep]
    \item Bapak Dr Eko Mulyanto Yuniarno, S.T.,M.T. dan Bapak Ahmad Zaini, S.T,M.T, selaku pembimbing skripsi, yang telah memberikan bimbingan, arahan, dan masukan yang berharga dalam penyusunan skripsi ini. Terima kasih atas kesabaran dan dedikasinya dalam membimbing penulis hingga mencapai titik ini.
    \item Ibu Dr. Diah Puspito Wulandari, S.T., M.Sc. dan Bapak Reza Fuad Rachmadi, S.T., M.T., Ph.D selaku dosen penguji yang telah memberikan masukan , arahan yang sangat berharga dalam penyusuna skripsi ini . Terima Kasih banyak atas kesabaran dan dedikasi sehingga bisa membantu penulis dalam menyelesaikan skripsi ini
    \item Keluarga Tercinta , Papa (Bayu Wibisono Muhammad Abduh) dan Mama (Mike Permanasari) yang senantiasa memberikan doa , dukungan , arahan , senantiasa memberikan dukungan baik dikala kondisi suka maupun duka sehingga bisa menyelesaikan perkuliahan ini dengan baik , hingga bisa mencapai titik ini 
    \item Om Yedi , Tante Nurul , Aldi , Eyang Harun yang berada di Surabaya maupun Keluarga Besar lain diluar Surabaya yang telah memberikan doa beserta dukungan sehingga bisa mencapai titik seperti ini
    \item Kedua Orang Tua Dafa Mahendra beserta Mbak Memes yang telah menerima saya dengan baik , mendukung dan mendoakan saya sehingga bisa mencapai titik sejauh ini
    \item Teman-Teman Seperjuangan di Departemen Teknik Komputer yang menjadi teman bermain , berdiskusi , belajar sehingga bisa mencapai titik seperti ini 
    \item Teman-Teman Grup History Discussion Democratic Republic (HDDR) , Baik itu Rio , Dafa Mahendra , Revan , Hanif Irbah , Matthew , Fachry , Naufal Afif , Naufal Safwan , Hilmy , Jiyad , Mas Reyhan dll yang sudah memberikan dukungan , doa , semangat , menjadi tempat curhat , berkeluh kesah dll sebagainya
    \item Grup Musik Favorit saya seperti baik itu Grup Idol Jepang seperti JKT48 , AKB48 , Nogizaka46 , Keyakizaka46 maupun Grup Musik Emo / Metal seperti Green Day , Metallica , Linkin Park , My Chemical Romance , Lyube , Paramore , Kaskad , Kino , blink-182 , Avenged Sevenfold dan grup musik lainnya yang tidak saya sebutkan satu persatu , Terima Kasih atas inspirasi dan motivasi dari setiap lirik yang dinyanyikan
    \item Semua pihak yang tidak bisa saya sebutkan satu persatu disini , saya mengucapkan terima kasih sebesar-besarnya
  
  \end{enumerate}
  
  Skripsi ini merupakan usaha penulis untuk memberikan kontribusi dalam bidang ilmu komputer, khususnya dalam menggabungkan teknologi MediaPipe dan Deep Learning untuk memprediksi pose semaphore. Penulis menyadari bahwa karya ini masih memiliki keterbatasan dan kekurangan. Oleh karena itu, penulis mengharapkan kritik, saran, dan masukan yang membangun untuk perbaikan di masa mendatang.

  Akhir kata, semoga skripsi ini dapat memberikan manfaat dan kontribusi yang positif bagi perkembangan ilmu pengetahuan dan teknologi. Penulis berharap bahwa hasil penelitian ini dapat menginspirasi dan membuka jalan bagi penelitian lebih lanjut dalam bidang yang sama.

Wassalamu'alaikum Wr. Wb.
  
  \begin{flushright}
    \begin{tabular}[b]{c}
      Surabaya, Juli 2023 \\
      \\
      \\
      \\
      \\
      Muhammad Daffa ZW
    \end{tabular}
  \end{flushright}